\section{实验题目}
编写一个简单模拟钢琴的程序,在屏幕上显示钢琴键位,通过键盘控制钢琴。

\section{问题分析与实验设计}

\begin{enumerate}
    \item 显示一段文字
    
    使用 INT 21H / AH = 09H 功能来输出字符串。
    
    利用数据区定义变量默认的连续性,可以简洁美观地定义多行字符串输出。
    
    \item 延迟 \label{delay}
    
    反复检查 61H 端口来实现延迟功能。每 $15.085 \mu s$,从 61H 端口输入的数据中的低5位会发生一次改变,监听这个改变来完成计时。
    
    如延迟 1ms 需要监测到 \textbf{约} 66 次这样的变化。
    
    可以封装一个以毫秒(ms)为单位的过程来处理这个延迟。
    
    \item 播放音乐
    
    通过将频率处理后发送至 42H 来使扬声器发声/禁声。
    
    首先要先将控制部分(端口号43H)初始化,然后再与 42H 端口进行通讯。
    
    \item 键盘输入
    
    使用 INT 16H / AH = 00H 功能来从键盘缓冲区读取一个键,无回显。
    
\end{enumerate}

\section{结果分析}

\lstinputlisting[caption={钢琴模拟屏幕显示}]{code/3/output.txt}

当按下屏幕上所示的键时,会发出对应音调的音符。

\addcontentsline{toc}{section}{附录:完整代码清单}
\section*{附录:完整代码清单}
\lstinputlisting[language={[x86masm]Assembler},caption={钢琴模拟主模块 main.asm}]{code/3/main.asm}
\lstinputlisting[language={[x86masm]Assembler},caption={钢琴模拟播放模块 play.asm}]{code/3/play.asm}
\lstinputlisting[language={[x86masm]Assembler},caption={钢琴模拟延迟模块 delay.asm}]{code/3/delay.asm}
