\section{实验题目}

求 100 以内能被 7 整除的所有正数之和。

\section{问题分析与实验设计}

解决这个问题我们有若干种思路:

\begin{enumerate}
    \item 穷举法
    
    100 以内的正数是一个有限的集合,因此可以穷举其中的所有元素,验证其是否能被 7 整除,如果是则加到累加器上。
    
    100 个数要做 100 次除法,过于暴力,效率太低,不予采纳。
    
    \item 递推法
    
    7 是最小的符合题设条件的数,设为$A_0$,以及从当前能被7整除的数$A_i$推出下一个能被7整除的数 $A_{i+1} = A_i + 7$。因此可以直接从 7 开始每次加 7,将不超过 100 的所有经过的数都加到累加器里即可。
    
    这个方法只需要涉及加法,但总的来说算法复杂度还是线性的 $O(n)$,不予采纳。
    
    \item 公式法
    
    由简单的数学推导得到一个更一般的结论:在区间$[0, a]$ 中,能被 $n$ 整除的整数之和为:
    
    $$
    \frac{n \lfloor \frac{a}{n} \rfloor (\lfloor \frac{a}{n} \rfloor + 1)}{2}
    $$
    
    代入 $a = 100, n = 7$ 可直接得出答案为 735。
    
    下取整除法在具体实现中比较简单,而且配合 DX, AX 可以组装出一个32位的乘法。
    
    \lstinputlisting[language={[x86masm]Assembler}, firstline=21, lastline=23, caption=精彩片段]{code/2/main.asm}
    
    这里巧妙地利用了循环移位来连接两个寄存器的数据。
    
\end{enumerate}

\section{结果分析}

\lstinputlisting[caption={DEBUG: 实验二运行结果}]{code/2/debug.txt}

当运行到 06CA:001E 时,程序进入将要结束的状态,此时 (DX AX) 的值是 000002DF,即16进制的735,答案正确。

\addcontentsline{toc}{section}{附录:完整代码清单}
\section*{附录:完整代码清单}
\lstinputlisting[language={[x86masm]Assembler},caption={求100以内能被7整除的数的和}]{code/2/main.asm}
