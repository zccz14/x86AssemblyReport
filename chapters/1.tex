\section{实验题目}

搭建汇编语言运行环境。

在屏幕上打印 “hello, world”。

\section{问题分析与实验设计}

\subsection{搭建环境}

本人使用Windows10系统,搭建汇编的开发环境需要如下组件:

\begin{itemize}
    
    \item 汇编器 Assembler
    
    Intel x86 汇编的汇编器可以使用\textbf{MASM \footnotemark},本人使用了代号为 \textbf{masm32v11r} 的版本。
    
    \footnotetext{MASM: Microsoft Macro Assembler 微软宏汇编器,但自MASM 6开始,MASM被集成到 VC 系列的工具链中不单独发布了。分离版本的MASM SDK 由一群爱好者维护升级,见 \href{http://masm32.com/}{http://masm32.com}}
    
    \item 链接器 Linker
    
    链接器被包含在上述 SDK 中,针对生成 16 位的可执行文件的要使用 link16.exe 。
    
    \item 运行环境 Runtime Environment
    
    DOSBOX \footnotemark 可以模拟 16 位的运行环境,本人使用其0.74版本。
    
    \footnotetext{DOSBOX: 可以在各种系统下模拟DOS运行环境,提供16位程序的运行环境,见 \href{www.dosbox.com}{http://www.dosbox.com}。}
    
    \item 调试器 Enhanced Debug
    
    使用 Debug \footnotemark 作为汇编程序的调试工具,本人使用其一个升级版本。
    
    \footnotetext{DEBUG: 著名的程序调试器,具有反汇编、单步调试等功能,见 \href{https://sites.google.com/site/pcdosretro/enhdebug}{https://sites.google.com/site/pcdosretro/enhdebug}}
    
\end{itemize}

\subsubsection{安装与配置}

将上述组件全部下载并使用默认设置安装:

结果汇编工具的执行目录在\textbf{C:\textbackslash masm32\textbackslash bin},而DOSBOX的执行目录在\textbf{C:\textbackslash Program Files (x86)\textbackslash DOSBox-0.74}。

并将DEBUG压缩包中的文件解压到\textbf{C:\textbackslash debug}。

然后将汇编工具的执行目录以及DOSBOX的执行目录添加到Windows系统的环境变量 PATH 中,打开命令行测试\textbf{ml} 与 \textbf{dosbox} 命令有效。

由于DOSBOX也是一个子系统,拥有自己的环境变量,需要将DEBUG的目录配置到DOSBOX的环境变量PATH中。

运行DOSBOX目录下的\textbf{DOSBox 0.74 Options.bat} 批处理文件,将打开一个配置文件,在文件最后的\textbf{autoexec \footnotemark} 域后面附加两行配置:

\footnotetext{autoexec 是DOSBOX启动后自动运行的脚本}

\begin{lstlisting}[caption=DOSBOX AutoExec 配置]
MOUNT D: C:\debug
SET PATH=D:\
\end{lstlisting}

表示将Windows中DEBUG所在的目录挂载到DOSBOX中的D盘分区,然后将D盘根目录添加到DOSBOX的环境变量PATH中。

成功后打开DOSBOX可以直接执行命令 \textbf{DEBUG} 来启动 DEBUG。

\subsubsection{用编辑器编写汇编代码}

新建一个文件夹,创建并编辑其中的 main.asm 文件,本人使用 \textbf{vim} 作为代码编辑器。

\begin{lstlisting}[language=bash]
$ mkdir 1
$ cd 1
$ vim main.asm
\end{lstlisting}

向 main.asm 中写入“hello, world” 程序代码:

\lstinputlisting[language={[x86masm]Assembler},caption={hello, world}]{code/1/main.asm}

\subsubsection{汇编、链接与运行}

编译与链接使用两个命令完成:

\begin{lstlisting}[language=bash,caption=汇编项目Build脚本——make.bat]
$ ml -c *.asm && link16 main.obj,main.exe,nul.map,.lib,nul.def
\end{lstlisting}

为了以后的模块化编程,也可以封装一个Build脚本——\textbf{make.bat}

在DOSBOX命令后附加可执行文件的名字即可运行:

\lstinline{$ dosbox main.exe}

\section{结果分析}

在DOSBOX下运行MAIN.EXE得到结果如下:

\begin{lstlisting}
C:\>MAIN.EXE
hello, world
\end{lstlisting}

查阅 INT 21H / AH = 09H 功能即可了解向控制台输出字符串的详细说明。